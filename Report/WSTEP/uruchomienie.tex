\section{Uruchomienie aplikacji}
Na płycie dołączonej do raportu znajdują się kompletne pliki źródłowe programu, oraz skompilowane wersje dla systemów Windows i Linux. Poniżej zamieszczono instrukcje uruchamiania obu wersji, oraz kompilacji.

\subsection{Windows}

\subsection{Linux}
Aby uruchomić aplikację pod systemem Linux, należy:
\begin{itemize}
\item zainstalować bibliotekę \emph{libfftw3} poleceniem \begin{verbatim}sudo apt-get install libfftw3,\end{verbatim}
\item zainstalować bibliotekę \emph{libqt4} poleceniem \begin{verbatim}sudo apt-get install libqt4.\end{verbatim}
\end{itemize}
Po zainstalowaniu wymaganych pakietów można uruchomić aplikację poprzez wykonanie skryptu \emph{run.sh}.

\subsection{Kompilacja plików źródłowych}
Do skompilowania plików źródłowych projektu wymagane są:
\begin{itemize}
\item biblioteka \emph{Qt 4.8.5},
\item biblioteka \emph{Qwt 6.0.2},
\item kompilator \emph{gcc 4.6.2} będący częścią środowiska \emph{MinGW},
\item środowisko programistyczne \emph{Qt Creator}.
\end{itemize}
Wszystkie biblioteki należy poprawnie zainstalować. W środowisku \emph{Qt Creator} należy zaimportować projekt i skonfigurować opcje budowania aplikacji.